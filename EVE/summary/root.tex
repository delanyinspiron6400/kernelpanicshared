%%%%%%%%%%%%%%%%%%%%%%%%%%%%%%%%%%%%%%%%%%%%%%%%%%%%%%%%%%%%%%%%%%%%%%%%%%%%%%%%
\documentclass[a4paper, 10 pt, conference]{ieeeconf}
\usepackage{stylesheet}


%-----------------------------------------------------------
%-
%-  Titel
%-
%-----------------------------------------------------------
\title{\LARGE \bf Entwurf von Echtzeitsystemen}

%-----------------------------------------------------------
%-
%-  Authors
%-
%-----------------------------------------------------------
\author{Martin Winter}

%-----------------------------------------------------------
%-
%-  References in references.bib
%-
%-----------------------------------------------------------
\usepackage[style=numeric,backend=bibtex, sorting=none]{biblatex}
\addbibresource{references.bib}

%%%%%%%%%%%%%%%%%%%%%%%%%%%%%%%%%%%%%%%%%%%%%%%%%%%%%%%%%%%%%%%%%%%%%%%%%%%%%%%%
\begin{document}

\maketitle
\thispagestyle{empty}
\pagestyle{empty}


%%%%%%%%%%%%%%%%%%%%%%%%%%%%%%%%%%%%%%%%%%%%%%%%%%%%%%%%%%%%%%%%%%%%%%%%%%%%%%%%
%-----------------------------------------------------------
%-
%-  abstract
%-
%-----------------------------------------------------------
\begin{abstract}

Das stellt einen Versuch dar, die Vorlesung "Entwurf von Echtzeitsystemen" sinnvoll zusammenzufassen, dies scheint mir schon jetzt als unmöglich und eigentlich hab ich überhaupt keine Lust, diese lustlos zusammengeklatschten Folien auch nur ein einziges Mal durchzulesen, aber vielleicht kann ich wenigstens dann mit ruhigem Gewissen diese Zusammenfassung lesen und diesen zusammen kopierten Müll, der sich Vorlesungsfolien nennt, nie mehr angreifen. \\

\end{abstract}


%%%%%%%%%%%%%%%%%%%%%%%%%%%%%%%%%%%%%%%%%%%%%%%%%%%%%%%%%%%%%%%%%%%%%%%%%%%%%%%%

%-----------------------------------------------------------
%-
%-  Einführung
%-
%-----------------------------------------------------------
\section{\textbf{Einführung}}
Wenn man von Echtzeitsystemen spricht, so ist meistens auch die Rede von \textbf{eingebetteten Systemen}, diese lassen sich folgendermaßen charakterisieren:

\begin{center}
\textit{Jedes System, das eine Form eines programmierbaren Computers enthält, jedoch nicht selbst ein Allzweck-Computer ist und Teil eines größeren Produkts ist, wird als eingebettetes System bezeichnet.}
\end{center}

Die Anwendungsbereiche erstrecken sich über die \textbf{Verkehrstechnik} (Automotive (ESP, diverse Steuergeräte, Motor-, Getriebesteuerung...), Avionik, Bahn und Schifffahrt), \textbf{Telekommunikation}, \textbf{Medizintechnik}, den \textbf{Consumer Elektronik Bereich}, \textbf{Industrielle Automation}, \textbf{Smart Buildings} und \textbf{Robotik}.\\
Echtzeitsysteme sind also oft auch eingebettete Systeme, der Begriff \textbf{Echtzeitverarbeitung} wird definiert als

\begin{center}
\textit{Die Betriebsart eines Rechensystems, bei der die Programme für die Verarbeitung von Daten, die von außen eintreffen, \textbf{permanent betriebsbereit sind}, so dass ihre Ergebnisse innerhalb einer \textbf{vorbestimmten Zeitperiode} zur Verfügung stehen, die Ankunftszeit der Daten können anwendungsabhängig \textbf{zufällig verteilt sein} oder \textbf{a priori} feststehen.}
\end{center} 

Somit kann man ein \textbf{Echtzeitsystem} wie folgt definieren

\begin{center}
\textit{System, dass explizite (begrenzte) \textbf{Antwortzeitbedingungen} einhalten muss, um nicht ernsthafte Konsequenzen zu riskieren, einschließlich einem Systemausfall.} 
\end{center}

Ein System gilt als \textbf{ausgefallen}, wenn es eine oder mehrere Anforderungen, die in den formalen Systemspezifikationen festgehalten sind, nicht erfüllt. Man kann dabei zwischen \textbf{harten} und \textbf{weichen Echtzeitsystemen} unterscheiden, bei ersteren führt 	eine Verletzung der Zeitbedingung unweigerlich zum Systemausfall, bei Zweiterem führt eine solche zu einer Leistungsminderung, aber nicht unbedingt zum Systemausfall. \\
\pagebreak
Man kann zwischen drei \textbf{Echtzeitbedingungen} unterscheiden
\begin{enumerate}
\item \textbf{weich}: Zeitschranke darf in gewissem Rahmen überschritten werden
\item \textbf{hart}: Zeitschranke muss auf jeden Fall eingehalten werden, sonst entsteht Schaden. 
\item \textbf{fest}: Aktion wird beim Überschreiten der Zeitschranke wertlos, aber es entsteht kein Schaden (weniger gebräuchlich).
\end{enumerate}

\subsection*{\textbf{Merkmale von Echtzeitsystemen}}
\subsubsection*{\textbf{Korrektheit}}
Funktionale und zeitliche Richtigkeit des Rechenergebnisses. \\
Bei Nicht-Echtzeitsystemen folgt aus logischer Korrektheit gleich die allgemeine Korrektheit, bei einem Echtzeitsystem muss zusätzlich noch eine zeitliche Korrektheit gelten, daher sind eigene Programmiertechniken erforderlich.
\subsubsection*{\textbf{Rechtzeitigkeit}}
Einhalten von Zeitbedingungen. \\
Ausgabedaten müssen rechtzeitig berechnet sein und zur Verfügung stehen, Eingabedaten müssen rechtzeitig abgeholt werden. Solche Zeitbedingungen werden durch den jeweiligen technischen Prozess festgelegt und sind somit anwendungsabhängig. \\
Man unterscheidet die Begriffe \textbf{Exakter Zeitpunkt}, \textbf{Deadline}, \textbf{Frühester Zeitpunkt}, \textbf{Zeitintervall}, \textbf{Periodisch}, \textbf{Aperiodisch}, \textbf{Absolut} und \textbf{Relativ}. 
\subsubsection*{\textbf{Gleichzeitigkeit}}
Erledigung von vielen parallelen Aufgaben unter Einhaltung von jeweils eigenen Zeitanforderungen. \\
Dabei wird ein \textbf{Echtzeit-Scheduler} zur Ablaufplanung benötigt. 
\subsubsection*{\textbf{Determiniertheit}}
Verhalten des Systems unter allen äußeren Bedingungen  \textbf{eindeutig} im Voraus bestimmbar(Für viele komplexe Systeme  eine zu strenge Forderung). Normalerweise genügt \textbf{Vorhersagbarkeit}, das heißt funktionales und zeitliches Verhalten muss in seiner \textbf{Wirkung} abschätzbar bleiben (z.B.: Die Auswirkung von Fehlern). 


\subsection*{\textbf{Prinzipieller Aufbau eingebetteter Systeme}}
Zentrum ist ein Computer, der als Mikrokontroller, als Parallelrechner oder ein räumlich verteiltes System, dieser kommuniziert mit dem Prozess-Interface und dem User-Interface. An das Prozess-Interface sind Sensoren und Aktoren angebunden und über das User-Interface können menschliche User mit dem System interagieren. Die Software (Betriebssystem, Anwendungsprogramme, Utilities...) muss den Echtzeitanforderungen für Echtzeitsysteme genügen. 
Standardsoftware ist meist nicht ohne weiteres einsetzbar. 

\pagebreak

\section{\textbf{Zielhardware}}





%%%%%%%%%%%%%%%%%%%%%%%%%%%%%%%%%%%%%%%%%%%%%%%%%%%%%%%%%%%%%%%%%%%%%%%%%%%%%%%%

%-----------------------------------------------------------
%-
%-  Appendix
%-
%-----------------------------------------------------------
\section*{APPENDIX}



%%%%%%%%%%%%%%%%%%%%%%%%%%%%%%%%%%%%%%%%%%%%%%%%%%%%%%%%%%%%%%%%%%%%%%%%%%%%%%%%
\printbibliography





\end{document}
