\documentclass[12pt,a4paper,ngerman]{article}
\usepackage{stylesheet}
\usepackage{epstopdf}
\begin{document}

\TUHeader                          %  Bitte Ausfüllen!!!
%----------------------------
{Digitale Übertragungssysteme}                       %  Übungstitel
%----------------------------
{28.10.2014}                        %  Übungsdatum
%----------------------------
{05}                            %  Gruppen-Nr.
%----------------------------
{Thomas Pichler}                   % Name des Protokollführers
%----------------------------
{
1.~Daniel Freßl, 1230028\\
2.~Thomas Neff, 1230319\\                    %  Übungsteilnehmer
3.~Thomas Pichler, 1230320 \\                   %  ...bei <4 Teilnehmer auskommentieren
4.~Martin Winter, 1130688\\
}
%----------------------------
{Assoc.-Prof. Dipl.-Ing. Dr. Klaus Witrisal}
{Katharina Ritt}                          %  Betreuer
%----------------------------
{Graz}                              %  Ort der Protokollerstellung
{\today}                            %  Datum Protokollerstellung




\pagebreak
  
\tableofcontents
  
\pagebreak

%-------------------------------------------------------------------------------
%
% Beginn des Protokolls
%
%-------------------------------------------------------------------------------

\section{Analyse eines unbekannten Filters}
\subsection{Aufgabenstellung}
In dieser Aufgabe soll das Verhalten eines unbekannten Filters untersucht werden. Zuerst soll eine grundsätzliche Aussage über die Art und Ordnung des Filters getroffen werden, danach soll das Filter so eingestellt werden, dass am Ausgang maximales Überschwingen beobachtet werden kann. 
Mit dieser Konfiguration soll nun die Eigenfrequenz bestimmt werden, das Bodediagramm aufgenommen werden und das System durch einen äquivalenten RLC-Serienschwingkreis ersetzt werden.

\subsection{Messaufbau}

\subsection{Tabellen}

\subsection{Formeln}

\subsection{Berechnungsbeispiele}

\subsection{Diagramme}

\subsection{Geräteliste}

\subsection{Diskussion}

\pagebreak
\section{Analyse eines RC-Tiefpass-Filters}
\subsection{Aufgabenstellung}
Für ein Filter mit den gegebenen Bauteilwerten $R = 13k\Omega$ und $C = 100 nF$ soll die Grenzfrequenz berechnet und gemessen werden, sowie das Bodediagramm aufgenommen werden.

\subsection{Messaufbau}

\subsection{Tabellen}

\subsection{Formeln}

\subsection{Berechnungsbeispiele}

\subsection{Diagramme}

\subsection{Diskussion}


\begin{thebibliography}{9}

\bibitem{skript}
  Teresa Meier, Dipl.-Ing. Georg Egger, Dipl.-Ing. Dr Michael Gebhart\\
  \emph{Übung C: RFID}\\
  Technische Universität Graz
\end{thebibliography}

 



   
\end{document}
